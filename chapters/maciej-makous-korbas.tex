\clearpage
\section{Maciej Korbaś}
\subsection{Nie ma dowodów, że to skopiowałem}
\begin{figure}[h]
    \centering
    \includegraphics[height=200pt]{pictures/makous/profile.png}
    \caption{Moje zdjęcie profilowe}
\end{figure}
\subsection{Jak nie używać tabel}
Czasami może kusić \underline{nas} aby używać tabel od przedstawiania macierzy. \large{To błąd!}\normalsize. Trzeba pamiętać, że mamy do tego \emph{specjalne znaczniki}. \textbf{Matematyka}, winna być zapisywana tak, jak zostało to przykazane

Ważnym jest, aby zwracać na to uwagę, gdyż może mieć to katastrofalne skutki dla naszego tekstu. Coś złoży się nie tak i wydawaca odrzuci naszą ciężką pracę, którą próbujemy zarobić na życie. Tak to czasami bywa i trudno coś z tym zrobić. Należy zatem pamiętać aby macierze formatować\\
\textbf{Nie tak:}\\\\
\begin{tabular}{|l|l|l|l|l|}
\hline
1 & 0 & 0 & 0 \\ \hline
0 & 1 & 0 & 0 \\ \hline
0 & 0 & 1 & 0 \\ \hline
0 & 0 & 0 & 1 \\ \hline
\end{tabular}\\\\
\textbf{Ale tak:}\\\\
$A = \begin{bmatrix}
1 & 0 & 0 & 0 \\ 
0 & 1 & 0 & 0 \\ 
0 & 0 & 1 & 0 \\ 
0 & 0 & 0 & 1 \\ 
\end{bmatrix}$
$2+2 \ne 5$

\subsection{Listy}
\begin{enumerate}
    \item Awizo
    \item Poprzez kuriera
    \item Impost
\end{enumerate}

\subsection{Niesegregowane listy}

\begin{itemize}
    \item[3] To nie jest pierwszy element
    \item Nie ma dowodów, że to drugi
    \item[0] Ten kiedyś był u góry
\end{itemize}

